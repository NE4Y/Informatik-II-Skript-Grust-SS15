\documentclass[a4paper, 20pt, openany]{book}
\usepackage{titlesec}
\titleformat{\chapter}[display]   
{\normalfont\huge\bfseries}{\chaptertitlename\ \thechapter}{20pt}{\Huge}   
\titlespacing*{\chapter}{5pt}{-50pt}{40pt}
\usepackage{ucs}
\usepackage[utf8x]{inputenc}
\usepackage{enumitem}
\usepackage{a4wide}
\usepackage{geometry}
\usepackage[automark]{scrpage2}
\pagestyle{scrheadings}
\usepackage{amssymb}
\usepackage{ulem}
\usepackage[doublespacing]{setspace}


\usepackage{titlesec}
\titleformat*{\section}{\large\bfseries}
\clearscrheadfoot
\ohead{Informatik II Skript - Steffen Lindner}
\cfoot{\pagemark}
\geometry{a4paper,left=10mm,right=10mm, top=10mm, bottom=2cm} 
\parindent0pt

\author{Steffen Lindner}
\title{\vspace{-2cm}Informatik II Skript}
\date{\today{}}

\begin{document}
\maketitle
\tableofcontents

\chapter{Einführung - 14.04.15}\uline{Scheme}: Ausdrücke, Auswertung und Abstraktion 

\uline{Dr.Racket}: Definitionsfenster (oberer Bereich), Interaktionsfenster (unterer Bereich) 

Die Anwendung von Funktionen wird in Scheme \uline{ausschließlich} in \uline{Präfixnotation} durchgeführt.

\paragraph{Beispiele}
\begin{flushleft}
	\begin{tabular}{c|c}
		Mathematik & Scheme \\
		44-2 & (- 44 2) \\
		f(x,y) & (f x y) \\
		$\sqrt{81}$& (sqrt 81) \\
		9² & (expt 9 2) \\
		3! & (! 3)  {}
	\end{tabular}
\end{flushleft}

Allgemein: ($<function> <arg1> <arg2> ...$)

(+ 40 2) und (odd? 42) sind Beispiele für \uline{Ausdrücke}, die bei \uline{Auswertung} einen Wert liefern. (Notation: $\rightsquigarrow$)

(+ 40 2) $\rightsquigarrow$ 42 ($\rightsquigarrow$ = Auswertng / Reduktion / Evalutation)

(odd? 42) $\rightsquigarrow$ \#f

Interaktionsfenster: Read $\rightarrow$ Eval $\rightarrow$ Print $\rightarrow$ Read ... (Read-Eval-Print-Loop aka. REPL)

\uline{Literale} stehen für einen konstanten Wert (auch konstante) und sind nicht weiter reduzierbar.

Literal: 

\#t, \#f (true, false, Wahrheitswerte) (boolean) \\
"abc", "x", " " (Zeichenkette) (String) \\
0 1904 42 -2 (ganze Zahlen) (Integer) \\
0.42 3.1415 (Fließkommazahl) (Reel) \\
1/2, 3/4 (rationale Zahl) (Rational) \\
$\backslash\_('')\_/"$ (Bilder) (Image)

\chapter{Ausdrücke, Defines, usw. - 16.04.2015}

Auswertung \uline{zusammengesetzter Ausdrücke} in mehreren Schritten (steps), von "innen nach außen" bis keine Reduktion mehr möglich ist.

\begin{center}
(+ (+ 20 20) (+ 1 1)) $\rightsquigarrow$ (+ 40 (+ 1 1) $\rightsquigarrow$ (+ 40 2) $\rightsquigarrow$ 42
\end{center}

\uline{Achtung}: Scheme rundet bei Arithmetik mit Fließkommazahlen (interne Darstellung ist binär).

Bsp.: Auswertung des zusammengesetzten Ausdrucks 0.7 + (1/2)/0.25 - 0.6/0.3

Arithmetik mit rationalen Zahlen ist exakt.

Ein Wert kann an einen \uline{Namen} (auch Identifier) gebunden werden , durch

\begin{center}
(define $<id>$ $<e>$) ($<id>$ Identifier, $<e>$ Expression)
\end{center}

Erlaubt konsistente Wiederverwendung und dient der Selbstdokumentation von Programmen.

\uline{Achtung}: Dies ist eine sogenannte \uline{Spezifikation} und kein Ausdruck. Insbesodnere besitzt diese Spezialform \uline{keinen} Wert, sondern einen Effekt: Name $<id>$ wird an den Wert von $<e>$ gebunden.



Namen können in Scheme fast beliebig gewählt werden, solange:

\begin{enumerate}
\item die Zeichen (kommt noch) nicht vorkommen
\item der Name nicht einem numerischen Literal gleicht
\item kein whitespace (Leerzeichen, Tabulatoren, Return) enthalten ist.
\end{enumerate}

Bsp.: euro $\rightarrow$ us$\$$

\uline{Achtung}: Groß-/Kleinschreibung ist in Identifiern nicht relevant.

Eine \uline{Lambda-Abstraktion} (auch: Funktion, Prozedur) erlaubt die Formulierung von Ausdrücken, die mittels \uline{Parametern} konkreten Werten abstrahieren:

\begin{center}
(lambda ($<p1> <p2> ...$) $<e>$), $<e>$ Rumpf
\end{center}

$<e>$ enthälft Vorkommen der Parameter $<p1>,<p2>$...

(lambda ...) ist eine Spezialform. Wert der Lambda-Abstraktion ist $\#<procedure>$

\uline{Anwendung} (auch: Applikation/Aufruf) der Lambda-Abstraktion führt zur Ersetzung der vorkommenden Parameter im Rumpf durch die angegebenen \uline{Argumente}: 

(lambda (days) (* days (* 155 min-in-a-day))) $\rightsquigarrow$ (* 365 (* 155 min-in-a-day)) $\rightsquigarrow$ 81468000

In Scheme leitet ein Semikolon einen \uline{Kommentar}, der bis zum Zeilenende reicht, ein und wird vom System bei der Auswertung ignoriert.

Prozeduren sollten im Programm eine ein-bis zweizeiliger Kurzberschreibung direkt voran gestellt werden.



\chapter{Signaturen, Testfälle - 21.04.15}
Eine \underline{Signatur} prüft, ob ein Name an einen Wert einer angegebenen Sorte (Typ) gebunden wird. Signaturverletzungen werden protokolliert.

\begin{center}
$(: \ <id>  \ <signatur>)$
\end{center}

Bereits eingebaute Signaturen:

\begin{itemize}
\item natural $\mathbb{N}$
\item integer $\mathbb{Z}$
\item rational $\mathbb{Q}$
\item real $\mathbb{R}$
\item number $\mathbb{C}$
\item boolean
\item string
\item image
\end{itemize}

(: ...) ist eine Spezialform ohne Wert, aber Effekt: Signaturprüfung \\

\underline{Prozedur-Signaturen} spezifizieren sowohl Signaturen für die Parameter p1, p2, ... , pn als auch den Ergebniswert der Prozedur:

\begin{center}
$(<signatur p_{1}> ... <signatur p_{n}> -> <signatur-ergebnis>)$
\end{center}

Prozedur-Signaturen werden \underline{bei jeder Anwendung} eine Prozedur auf Verletzung geprüft. \\


\underline{Testfälle} dokumentieren das erwartete Ergebnis einer Prozedur für ausgewählte Argumente:

\begin{center}
$(check-expect <e_{1}> <e_{2}>)$
\end{center}

Werte Ausruck $<e_{1}>$ aus und teste, ob der erhaltene Wert der Erwarung (= der Wert von $<e_{2}>$) entspricht. \\

Einer Prozedurdefinition sollten Testfälle direkt vorangestellt werden.

Spezialform: Kein Wert, aber Effekt: Testverletzung protokollieren.\\

\underline{Konstruktionsanleitung für Prozeduren}
\begin{itemize}
\item ; ... (1) Kurzbeschreibung (1-2 zeiliger Kommentar mit Bezug auf Parameter)
\item (: ...) (2) Signatur
\item (check-expect ...) (3) Testfälle
\item (define (lambda (...) ...) (4) Prozedur + Rumpf
\end{itemize}

\underline{Top-Down-Entwurf} (Programmieren durch "Wunschdenken") 

Bsp.: Zeichen Ziffernblatt (Stunden- und Minutenzeiger) zur Uhrzeit H:m auf einer analogen 24h-Uhr 
\begin{itemize}
\item Minutenzeiger legt 360°/60 pro Minute zurück (360/60 * m)
\item Stundenzeiger legt 360°/12 pro Stunde zurück (360/12 * h + 360/12 * m/60)
\end{itemize}

\chapter{Substitutionsmodell, Fallunterscheidung - 23.04.15}
\underline{Reduktionsregeln} für Scheme (Fallunterscheidug je nach Ausrucksart)

Wiederhole, bis keine Reduktion mehr möglich:
\begin{itemize}
\item Literal (1, "abc", \#t, ...) [eval$_{lit}$]

l $\rightsquigarrow$ l
\item Identifier id (pi, clock-face, ...) [eval$_{id}$]

id $\rightsquigarrow$ gebundener Wert
\item Lambd-Abstraktion 

(lambda () ) $\rightsquigarrow$ (lambda () ) [eval$_{\lambda}$]

\item Applikation (f, e1, e2)
 \begin{itemize}
 \item (1) f, e1, e2 reduziere, erhalte f', e1', e2' 
 \item (2)
   \begin{itemize}
   \item Operation f' auf e1', e2', ... falls f' primitive Operation (+, *, ...) [apply$_{prim}$]
   \item Argumentenwert e1', e2', ... Rumpf von f' einsetzen, dann Rumpf reduzieren , falls f' Lambdaabstraktion [apply$_{\lambda}$]
   \end{itemize}
 \end{itemize}
\end{itemize}

\underline{Beispiel:} Applikation

(+ 40 2)

$\rightsquigarrow$ (\#$<$ procedure + $>$ 40 2) $\rightsquigarrow$ 42

eval$_{lit}$ (+)

eval$_{lit}$ (40)

eval$_{lit}$ (2)\textsl{•}


(position-minute-hand 30)

$\rightsquigarrow$ ((lambda (m) (* degrees-per-minute m)) 30)

$\rightsquigarrow$ (* degrees-per-minute 30)

$\rightsquigarrow$ (* degress-per-minute 30)

$\rightsquigarrow$ (\#$<$procedure * $>$ 360/60 30)

Bezeichnen (lambda (x) (* x x )) und (lambda (r) (* r r)) die gleiche Prozedur? $\Rightarrow$ Ja!

\underline{Achtung}: Das hat Einfluss auf das korrekte Einsetzen von Argumenten für Parameter! (s. apply$_{\lambda}$)

Das \underline{bindenen Vorkommen} eines Identifiers x kann im Programmtext systematisch bestimmt werden: suche strik von "innen nach außen" bis zum ersten 

\begin{itemize}
\item (lambda (x) )
\item (define x )
\end{itemize}

(Prinzip der \underline{lexikalischen Bindung})

Übliche Notation in der Mathematik: \underline{Fallunterscheidung}

$ maximum(x_{1}, x_{2})=\left\{\begin{array}{cl} x_{1}, & \mbox{falls } x_{1} \geq x_{2}\\ x_{2}, & \mbox{sonst} \end{array}\right. $

\underline{Tests} auch (Prädikate) sind Funktionen, die einen Wert der Signatur boolean liefern. Typische primitive Tests:

\begin{itemize}
\item (: = (number number $\rightarrow$ boolean))
\item (: $<$ (real real $\rightarrow$ boolean)), auch $>, \le, \geq$
\item (: string=? (string string $\rightarrow$ boolean)), auch string$>$?, string$\le$?
\item (: boolean? (boolean boolean $\rightarrow$ boolean))
\item (: zero? (number $\rightarrow$ boolean))
\item odd?, even?, positive?, negative?, ...
\end{itemize}

Binäre Fallunterscheidung: \underline{if}

(if $<t_{1}> <e_{1}> <e_{2}>$) \\

Mathematisch: 
$ \left\{\begin{array}{cl} e_{1}, & \mbox{falls } t_{1}\\ e_{2}, & \mbox{sonst} \end{array}\right. $

\chapter{One-of Signatur - 28.04.15}
Die Signatur \underline{one-of} lässt genau einen der aufgezählten n Werte zu: 
\begin{center}
	(one-of $<e_1> ... <e_n>$)
\end{center}

Reduktion von if: \\

(if $t_{1}\  e_{1}  \ e_{2}$) $ \rightsquigarrow$ 
$ \left\{\begin{array}{cl} <e_{1}>, & \mbox{falls t1' = \#t ; e2 wird niemals ausgewertet} \\ <e_{2}>, & \mbox{sonst; e1 wird niemals ausgewertet} \end{array}\right. $ \\



(1) Reduziere $t_{1}$, erhalte $t_{1}'$ \\

Spezialform Fallunterscheidung (conditional expression):
\begin{center}
(cond ($<t_1 > <e_1 >$) ... ($<t_n > <e_n >$) (else $<e_{n+1}>$)) (else optional)
\end{center}

Werte die Tests in der Reihenfolge $t_{1},t_{2},...,t_{n}$ aus. Sobald $t_i $ \# t ergibt werte Zweig $e_i $ aus. $e_i $ ist das Ergebnis der Fallunterscheidung. Wenn $t_n $ \# f liefert, dann liefere
\begin{center}
$ \left\{\begin{array}{cl} Fehlermeldung, & \mbox{"cond: alle Tests ergaben \#f", falls kein else-Zweig \ } \\ <e_{n+1}>, & \mbox{sonst} \end{array}\right. $
\end{center}

Reduktion von cond [eval$_{cond}$]

\begin{center}
	(cond ($<t_{1}> <e_{1}>) (<t_2> <e_2>) ...$) $\rightsquigarrow    \left\{\begin{array}{cl} <e_{1}>, & \mbox{falls t1' = \#f} \\ (cond (<t_{2}> <e_{2}> ) (...) ), & \mbox{sonst} \end{array}\right. $
\end{center}

Reduziere $t_1$, erhalte $t_1$'.

(cond ) $\rightsquigarrow$ Fehlermeldung "Alle Tests..."

(cond (else $<e_{n+1}>$)) $\rightsquigarrow$ $e_{n+1}$

cond ist "systematischer Zucker" \\
(auch: abgleitete Form) für eine verschachtelte Anwendung von 'if':

(cond ($<t_1> <e_1>$) ($<t_1> <e_1>$) ... ))) entspricht (if $<t_1> <e_1> (if <t_1> <e_1> (if ...)$)

Spezialformen 'and' und 'or':
\begin{center}
$(or <t_!> <t_2> ... <t_n>)$ entspricht $(if <t_1> \#t (or <t_2> ...)$)

(or) $\rightsquigarrow$ \#f
\end{center}
\begin{center}
$(and <t_1>...<t_n> \rightsquigarrow (if <t_1> (and <t_2> ... <t_n>) \#f$)

(and) $\rightsquigarrow \# t$
\end{center}

\chapter{Zusammengesetzte Daten - 30.04.15}
Ein Charakter \underline{besteht aus} drei \underline{Komponenten}.
\begin{itemize}
  \item Name des Charakters (name)
  \item Handelt es sich um einen Jedi? (jedi?)
  \item Stärke der Macht (force)
\end{itemize}

$\rightarrow$ \underline{Datendefinition} für zusammengesetzte Daten.

Konkreter Charakter: \\

\begin{tabular}{c|c}
Name & "Luke Skywalker" \\ \hline
jedi? & \#f \\ \hline
force & 25 \\
\end{tabular}\\

; Ein Charakter (character) besteht aus \\
; - Name (name)\\
; - Jedi-Status (jedi?)\\
; - Stärke der Macht (force)\\

\begin{center}
(define-records-procedures charakter

make-character

character?

(character-name

character-jedi

character-force))
\end{center}

(make-character n j f) $\rightsquigarrow$ $<records>$ (konstruktion)

(character-name $<record>$ $\rightsquigarrow$ n (Komponentenzugriff)

(character-jedi? $<record>$) $\rightsquigarrow$ j (Komponentenzugriff)

(character-force $<record>$) $\rightsquigarrow$ f (Komponentenzugriff)

Zusammengesetzte Daten = \underline{Records} in Scheme.

Record-Definition legt fest:

\begin{itemize}
  \item Record-Signatur
  \item \underline{Konstruktor} (Baut aus Komponenten einen Record)
  \item Prädikat (liegt Record vor?)
  \item Liste von \underline{Selektoren} (lesen jewils eine Komponenten des Records)
\end{itemize}

\begin{center}
(define-records-procedures $<t>$

make-$<t>$ ; Konstruktor

$<t>$?

($<t>-<comp_1> ...<t>-<comp_n>4))$ ; Liste der n-Seleketoren
\end{center}

Verträge des Konstruktors / der Selektoren für Record-Signatur $<t>$ mit n Komponenten namens $<comp_1> ... <comp_n>$:

\begin{itemize}
  \item (: make-$<t>$ ($<t_1>...<t_n> \rightarrow <t>$))
  \item (: $<t>-<comp_1> (<t> \rightarrow <t_1>$))
  \item ...
  \item (: $<t>-<comp_n> (<t> \rightarrow <t_n>$))
\end{itemize}

Es gilt für die Strings n, Booleans j und Integer f:

(character-name (make-character n j f)) = n 

(analog für den Rest)

Interaktion von Funktionen (\underline{algebraische Eigenschaften}).

Spezialform \underline{check-property}:

\begin{center}
(check-property

(for-all (( $<id_1> \ <sig_1> ... <id_n> \ <sig_n>$))

$<e>$))
\end{center}
$<e>$ bezieht sich auf $<id_1>... <id_n>$.

Test erfolgreich, falls $<e>$ für bel. gewählte Bindungen für $<id_1>...<id_n>$ immer \#t ergibt.

Interaktion von Selektor und Konstruktor:

\begin{center}
  (check-property
  
  (for-all ((n string)
  
  (j booleans)
  
  (f integer))
  
  (string=? (character-name (make-character n j f)) n )))
\end{center}

\underline{Beispiel:} Die Summe zweier natürlicher Zahlen ist mindestens so groß wie jede dieser Zahlen: $\forall x_1, x_2 \in \mathbb{N} : x_1 + x_2 \geq max \ {x_1, x_2}$

\begin{center}
(check-property

(for-all (($x_1$ natural)

($x_2$ natural))

($\geq$ (+ $x_1$ $x_2$) (max $x_1$ $x_2$))))
\end{center}

Konstruktion von Funktionen, die zusammengesetzte Daten \underline{konsumieren}:

\begin{itemize}
  \item Welche Record-Komponenten sind relevant für Funktionen?
  
  $\rightarrow$ Schablone:
  
  ; könnte Charakter e ein Sith-Lord sein?
  
  (: sith? (character $\rightarrow$ boolean))
  
  (define sith?
  
  (lambda (e) ... (character-jedi? c) ... (character-force c) ... ))
\end{itemize}

Konstrukton von Funktionen, die zusammengesetzte Daten \underline{konstruieren}: 

\begin{itemize}
  \item Der Konstruktor \underline{muss} aufgerufen werden.
  
  $\rightarrow$ Schablone:
  
  (define 
  
  (lambda (...)
  
  ... (make-$<t>$ ...) ...))

\end{itemize}



\end{document}