\documentclass[a4paper, 20pt, openany]{book}
\usepackage{ucs}
\usepackage[utf8x]{inputenc}
\usepackage{enumitem}
\usepackage{a4wide}
\usepackage{geometry}
\usepackage[automark]{scrpage2}
\pagestyle{scrheadings}
\usepackage{amssymb}
\usepackage{ulem}
\usepackage[doublespacing]{setspace}

\usepackage{titlesec}
\titleformat*{\section}{\large\bfseries}
\clearscrheadfoot
\ohead{Informatik II Skript - Steffen Lindner}
\cfoot{\pagemark}
\geometry{a4paper,left=10mm,right=10mm, top=10mm, bottom=2cm} 
\parindent0pt

\author{Steffen Lindner}
\title{\vspace{-2cm}Informatik II Skript}
\date{\today{}}

\begin{document}
\maketitle
\tableofcontents

\chapter{Einführung - 14.04.15}\uline{Scheme}: Ausdrücke, Auswertung und Abstraktion 

\uline{Dr.Racket}: Definitionsfenster (oberer Bereich), Interaktionsfenster (unterer Bereich) 

Die Anwendung von Funktionen wird in Scheme \uline{ausschließlich} in \uline{Präfixnotation} durchgeführt.

\paragraph{Beispiele}
\begin{flushleft}
	\begin{tabular}{c|c}
		Mathematik & Scheme \\
		44-2 & (- 44 2) \\
		f(x,y) & (f x y) \\
		$\sqrt{81}$& (sqrt 81) \\
		9² & (expt 9 2) \\
		3! & (! 3)  {}
	\end{tabular}
\end{flushleft}

Allgemein: ($<function> <arg1> <arg2> ...$)

(+ 40 2) und (odd? 42) sind Beispiele für \uline{Ausdrücke}, die bei \uline{Auswertung} einen Wert liefern. (Notation: $\rightsquigarrow$)

(+ 40 2) $\rightsquigarrow$ 42 ($\rightsquigarrow$ = Auswertng / Reduktion / Evalutation)

(odd? 42) $\rightsquigarrow$ \#f

Interaktionsfenster: Read $\rightarrow$ Eval $\rightarrow$ Print $\rightarrow$ Read ... (Read-Eval-Print-Loop aka. REPL)

\uline{Literale} stehen für einen konstanten Wert (auch konstante) und sind nicht weiter reduzierbar.

Literal: 

\#t, \#f (true, false, Wahrheitswerte) (boolean) \\
"abc", "x", " " (Zeichenkette) (String) \\
0 1904 42 -2 (ganze Zahlen) (Integer) \\
0.42 3.1415 (Fließkommazahl) (Reel) \\
1/2, 3/4 (rationale Zahl) (Rational) \\
$\backslash\_('')\_/"$ (Bilder) (Image)

\chapter{Ausdrücke, Defines, usw. - 16.04.2015}

Auswertung \uline{zusammengesetzter Ausdrücke} in mehreren Schritten (steps), von "innen nach außen" bis keine Reduktion mehr möglich ist.

\begin{center}
(+ (+ 20 20) (+ 1 1)) $\rightsquigarrow$ (+ 40 (+ 1 1) $\rightsquigarrow$ (+ 40 2) $\rightsquigarrow$ 42
\end{center}

\uline{Achtung}: Scheme rundet bei Arithmetik mit Fließkommazahlen (interne Darstellung ist binär).

Bsp.: Auswertung des zusammengesetzten Ausdrucks 0.7 + (1/2)/0.25 - 0.6/0.3

Arithmetik mit rationalen Zahlen ist exakt.

Ein Wert kann an einen \uline{Namen} (auch Identifier) gebunden werden , durch

\begin{center}
(define $<id>$ $<e>$) ($<id>$ Identifier, $<e>$ Expression)
\end{center}

Erlaubt konsistente Wiederverwendung und dient der Selbstdokumentation von Programmen.

\uline{Achtung}: Dies ist eine sogenannte \uline{Spezifikation} und kein Ausdruck. Insbesodnere besitzt diese Spezialform \uline{keinen} Wert, sondern einen Effekt: Name $<id>$ wird an den Wert von $<e>$ gebunden.



Namen können in Scheme fast beliebig gewählt werden, solange:

\begin{enumerate}
\item die Zeichen (kommt noch) nicht vorkommen
\item der Name nicht einem numerischen Literal gleicht
\item kein whitespace (Leerzeichen, Tabulatoren, Return) enthalten ist.
\end{enumerate}

Bsp.: euro $\rightarrow$ us$\$$

\uline{Achtung}: Groß-/Kleinschreibung ist in Identifiern nicht relevant.

Eine \uline{Lambda-Abstraktion} (auch: Funktion, Prozedur) erlaubt die Formulierung von Ausdrücken, die mittels \uline{Parametern} konkreten Werten abstrahieren:

\begin{center}
(lambda ($<p1> <p2> ...$) $<e>$), $<e>$ Rumpf
\end{center}

$<e>$ enthälft Vorkommen der Parameter $<p1>,<p2>$...

(lambda ...) ist eine Spezialform. Wert der Lambda-Abstraktion ist $\#<procedure>$

\uline{Anwendung} (auch: Applikation/Aufruf) der Lambda-Abstraktion führt zur Ersetzung der vorkommenden Parameter im Rumpf durch die angegebenen \uline{Argumente}: 

(lambda (days) (* days (* 155 min-in-a-day))) $\rightsquigarrow$ (* 365 (* 155 min-in-a-day)) $\rightsquigarrow$ 81468000

In Scheme leitet ein Semikolon einen \uline{Kommentar}, der bis zum Zeilenende reicht, ein und wird vom System bei der Auswertung ignoriert.

Prozeduren sollten im Programm eine ein-bis zweizeiliger Kurzberschreibung direkt voran gestellt werden.

\end{document}